%\documentclass[varwidth]{standalone}
\documentclass[10pt]{amsart}
\usepackage{amscd,amsxtra,color,amsthm}

\usepackage[all]{xy}
\usepackage{etex}
\usepackage{pictex}
\usepackage{graphicx}
\usepackage{tikz}
\usepackage[utf8]{inputenc} 
\usepackage[T1]{fontenc}
\usepackage[all]{xy}
\usepackage{etex}
\usepackage{pictex}
\usepackage{graphicx}
\usepackage{mathtools}
\DeclarePairedDelimiter{\ceil}{\lceil}{\rceil}
\DeclarePairedDelimiter{\floor}{\lfloor}{\rfloor}
\usepackage{comment}

\textheight=9in \textwidth=6.2in \topmargin=0in
\oddsidemargin=.15in \evensidemargin=.15in

\begin{document}
\parskip10pt
\parindent12pt
\baselineskip16pt

%%%%%%%%%%%%%%%%%%%%%%%%%%%%%%%%%%%%%%%%%%%%%%%%%%%%%%%%%%%%%%%%%%%%%%%%%%%%%%%%%%%%%%%%%%%%%%%%
%%  Definitions
%%%%%%%%%%%%%%%%%%%%%%%%%%%%%%%%%%%%%%%%%%%%%%%%%%%%%%%%%%%%%%%%%%%%%%%%%%%%%%%%%%%%%%%%%%%%%%%%

\def\G{\widetilde{G}}
\def\B{\widetilde{B}}
\def\T{\widetilde{T}}
\def\C{\mathbb{C}}
\def\A{\mathbb{A}}
\def\Z{\mathbb{Z}}
\def\R{\mathbb{R}}
\def\Q{\mathbb{Q}}
\def\N{\mathbb{N}}
\def\C{\mathbb{C}}
\def\F{\mathbb{F}}
\def\I{\mathbb{I}}
\def\H{\mathcal{H}}
\def\e{\varepsilon}
\def\s{\underline s}
\def\z{\zeta }
\def\vp{\varpi }
\def\O{\mathcal O}
\def\v{\upsilon }
\def\U{\Upsilon }
\def\p{\wp }
\def\p{\mathfrak{p}}
\def\B{\mathfrak{B}}

\newtheorem{theorem}{Theorem}%[section]
\newtheorem{lemma}[theorem]{Lemma}

%%%%%%%%%%%%%%%%%%%%%%%%%%%%%%%%%%%%%%%%%%%%%%%%%%%%%%%%%%%%%%%%%%%%%%%%%%%%%%%%%%%%%%%%%%%%%%%%
%%  Title Page
%%%%%%%%%%%%%%%%%%%%%%%%%%%%%%%%%%%%%%%%%%%%%%%%%%%%%%%%%%%%%%%%%%%%%%%%%%%%%%%%%%%%%%%%%%%%%%%%

\title{Byzantine Generals Problem}

\author{Graham Swain}

\begin{abstract}
    In this paper we explore how many guards are needed to guard a museum. We construct a lemma 
    showing how to triangulate a convex polygon. We then use that and vertex coloring to develop 
    a proof by induction to find the amount of guards that could sufficiently guard a museum.
\end{abstract}

\maketitle

%%%%%%%%%%%%%%%%%%%%%%%%%%%%%%%%%%%%%%%%%%%%%%%%%%%%%%%%%%%%%%%%%%%%%%%%%%%%%%%%%%%%%%%%%%%%%%%%
%%  Introduction (Definitions)
%%%%%%%%%%%%%%%%%%%%%%%%%%%%%%%%%%%%%%%%%%%%%%%%%%%%%%%%%%%%%%%%%%%%%%%%%%%%%%%%%%%%%%%%%%%%%%%%

\section{Introduction}
The following problem was coined by Victor Klee in 1973. 
Suppose the manager of a museum wants to place guards to watch every point of the building at all times. 
Guards are stationed at fixed points, and they have the ability to turn around. 
How many guards are needed to achieve this? Before we dive into the problem, let's introduce some preliminary definitions and corresponding examples.

\noindent {\bf Definition.} \emph{A polygon is called \textbf{convex} when all of its interior angles 
are less than $180^\circ$.}
\begin{figure}[h!]
    \centering
    \includegraphics[scale=.25]{figures/convex-polygon.pdf}
    \includegraphics[scale=.2]{figures/non-convex-polygon.pdf}
    \caption{Convex polygon on the left, non-convex polygon on the right.}
\end{figure}

\noindent {\bf Definition.} \emph{A graph is called \textbf{planar} when all of its vertices and edges 
are contained in a single plane.}
\begin{figure}[h!]
    \centering
    \includegraphics[scale=.25]{figures/planar-graph.pdf}
    \caption{An example of a planar graph.}
\end{figure}

\pagebreak

% Needs work
\noindent {\bf Definition.} \emph{A planar graph $G$ is said to be \textbf{triangulated} when adding 
another edge to $G$ results in a nonplanar graph.}
\begin{figure}[h!]
    \centering
    \includegraphics[scale=.25]{figures/triangulation.pdf}
    \caption{An example of a triangulated graph.}
\end{figure}

Note that this definition is equivalent to \textbf{maximal planar}. Some sources will 
state a graph is triangulated when no straight edges can be added, an example of that would be a 
square with a single diagonal; you could draw a new edge connecting the two non-connected 
vertices, but the edge would not be straight. For this paper, we are only concerned about the 
interior of the polygons. Therefore, we will consider a polygon triangulated when adding 
an edge that is contained within the polygon will make it nonplanar.

\noindent {\bf Definition.} \emph{ A proper \textbf{vertex coloring} of a graph $G$ is an assignment of 
colors to the vertices of $G$ such that no two adjacent vertices receive the same color.}
\begin{figure}[h!]
    \centering
    \includegraphics[scale=.25]{figures/coloring.pdf}
    \caption{This is a 3-coloring of G.}
\end{figure}

% \noindent {\bf Definition.} \emph{}

% \pagebreak

%%%%%%%%%%%%%%%%%%%%%%%%%%%%%%%%%%%%%%%%%%%%%%%%%%%%%%%%%%%%%%%%%%%%%%%%%%%%%%%%%%%%%%%%%%%%%%%%
%%  Introduction (Context)
%%%%%%%%%%%%%%%%%%%%%%%%%%%%%%%%%%%%%%%%%%%%%%%%%%%%%%%%%%%%%%%%%%%%%%%%%%%%%%%%%%%%%%%%%%%%%%%%
% transition?
Coming back to the question we originally introduced, we will imagine the walls of the museum are a polygon consisting of $n$ sides. 
If the polygon is convex, then we only need one guard to oversee the entire museum. 
\begin{figure}[h!]
    \centering
    \includegraphics[scale=.4]{figures/convex-museum.pdf}
\end{figure}

However, the walls of the museum can take on the shape of {\bf any} closed polygon. 
\begin{figure}[h!]
    \centering
    \includegraphics[scale=.14]{figures/floor-plan.png}
    \caption{Floor layout of the second floor of the Met. (From NY Times, 2017)}
\end{figure}

Let's consider a comb-shaped museum with $n = 3m$ walls.
\begin{figure}[h!]
    \centering
    \includegraphics[scale=.15]{figures/comb-shape.pdf}
\end{figure}

Notice that point 1 can only be seen by a guard stationed anywhere in the shaded triangle containing the point. 
This applies to the other points 2, 3,..., $m$. 
We can see that this requires {\bf at least} $m = \frac{n}{3}$ guards, one for each shaded triangle. 

However, $m$ guards are also sufficient. 
Since the guards can be placed at the bottom of the triangles, and 
each guard can watch their own triangle as well as down the hall.
Thus, the floor of $\frac{n}{3}$ guards are needed for an $n$-walled museum.

%%%%%%%%%%%%%%%%%%%%%%%%%%%%%%%%%%%%%%%%%%%%%%%%%%%%%%%%%%%%%%%%%%%%%%%%%%%%%%%%%%%%%%%%%%%%%%%%
%%  Triangulations
%%%%%%%%%%%%%%%%%%%%%%%%%%%%%%%%%%%%%%%%%%%%%%%%%%%%%%%%%%%%%%%%%%%%%%%%%%%%%%%%%%%%%%%%%%%%%%%%
\section{Triangulation}

Before we get into the main proof of our paper, we'd like to go over a proof of a lemma we will 
use. This lemma proves that there exists a triangulation for any non-convex polygon.
% \begin{figure}
%     \centering
%     \includegraphics[scale=.35]{figures/triangulation-proof-heading.pdf}
% \end{figure}

We will look at the triangulation of a pentagon to show the basic process of triangulating a polygon.
Draw a diagonal between any two vertices and then split the pentagon at the diagonal. The result will
be a triangle and a quadrilateral. We then want to draw a diagonal between any two points of the
quadrilateral, and then split it on the diagonal. The result will be two more triangles and a 
triangulated pentagon.
\begin{figure}[h!]
    \centering
    \includegraphics[scale=.17]{figures/triangulate-pentagon-split-fb-dos.pdf}
    \caption{This shows the process of triangulating a pentagon.}
\end{figure}
% look at feedback

\begin{proof} \
    Let $P$ be a non-convex polygon with $n$ sides. We will be using induction for this proof. 
    Our base case is $n = 3$, which is a triangle. 
    For our inductive hypothesis, assume $P$ can be triangulated if $n$ is less than $k$, $k$ being 
    any integer greater than 3. 
    We will show that $P$ can be triangulated when $n = k$.
    \begin{figure}[h!]
        \centering
        \includegraphics[scale=.2]{figures/triangulation-proof.pdf}
    \end{figure}

    \vspace{-20pt}
    To prove this, we will show that there exists a diagonal that can split $P$ into 
    two smaller polygons that can be triangulated. 
    A vertex $A$ is convex if its interior angle is less than $180^\circ$. We know that 
    the sum of the interior angles of a polygon is $(n-2)180$.  
    If we imagine that there are $n$ vertices being placed into $n-2$ boxes of $180^\circ$, 
    then by the pigeonhole principle, there must be a convex vertex $A$. 
    Looking at the two neighbors $B$ and $C$ of $A$. 
    If the segment $BC$ is entirely in $P$, then we have our diagonal. 
    If the segment is not entirely in $P$, then the triangle $ABC$ contains other vertices. 

    \begin{figure}[h!]
        \centering
        \includegraphics[scale=.2]{figures/triangulation-proof-dotted.pdf}
    \end{figure}

    Slide $BC$ towards $A$ until you hit the last vertex $Z$ in $ABC$. 
    Since $AZ$ is within $P$, we have our diagonal. 

    \begin{figure}[h!]
        \begin{minipage}{.5\textwidth}
            \centering
            \includegraphics[scale=.2]{figures/triangulation-proof-diagonal.pdf}            
        \end{minipage}%
        \begin{minipage}{.5\textwidth}
            \centering
            \includegraphics[scale=.2]{figures/triangulation-proof-split.pdf}            
        \end{minipage}
    \end{figure}

    Now that we have our diagonal we can split $P$ into smaller polygons, both with side $AZ$, 
    with less than $k$ sides. 
    By our inductive hypothesis, both of these polygons can be triangulated. 
    Thus $P$ can be triangulated.
\end{proof}

%%%%%%%%%%%%%%%%%%%%%%%%%%%%%%%%%%%%%%%%%%%%%%%%%%%%%%%%%%%%%%%%%%%%%%%%%%%%%%%%%%%%%%%%%%%%%%%%
%%  Theorem
%%%%%%%%%%%%%%%%%%%%%%%%%%%%%%%%%%%%%%%%%%%%%%%%%%%%%%%%%%%%%%%%%%%%%%%%%%%%%%%%%%%%%%%%%%%%%%%%

\section{Art Gallery Theorem}

Now that we know that any polygon can be triangulated we can continue on to our main proof.

\begin{theorem}[Art Gallery Theorem] \
    For any museum with $n$ walls, the floor of $\frac{n}{3}$ guards suffice.
\end{theorem}

%%%%%%%%%%%%%%%%%%%%%%%%%%%%%%%%%%%%%%%%%%%%%%%%%%%%%%%%%%%%%%%%%%%%%%%%%%%%%%%%%%%%%%%%%%%%%%%%
%%  Proof
%%%%%%%%%%%%%%%%%%%%%%%%%%%%%%%%%%%%%%%%%%%%%%%%%%%%%%%%%%%%%%%%%%%%%%%%%%%%%%%%%%%%%%%%%%%%%%%%

\begin{proof} \
    Let $P$ be a polygon with $n$ walls. 
    Because of our lemma, we know we can triangulate $P$. 
    You can think of $P$ as a planar graph with the corners as vertices and the walls and 
    diagonals as edges. 
    \begin{figure}[h!]
        \centering
        \includegraphics[scale=.25]{figures/art-gallery-triangulated.pdf}
    \end{figure}

    We will now show that $P$ is 3-colorable. 
    For $n=3$, the coloring trivial, as it would be a triangle and  every vertex would be a 
    different color. 
    For $n>3$, pick any two vertices $u$ and $v$ connected by a diagonal and split $P$ along $uv$. 
    \begin{figure}[h!]
        \centering
        \includegraphics[scale=.25]{figures/art-gallery-triangulated-fb-1b.pdf}
    \end{figure}

    This will give us two smaller triangulated graphs that both contain the edge $uv$. 
    Continue in this manner until only triangles remain. 
    \begin{figure}[h!]
        \begin{minipage}{.3\textwidth}
            \centering
            \includegraphics[scale=.17]{figures/art-gallery-triangulated-fb-2b.pdf}          
        \end{minipage}%
        \begin{minipage}{.3\textwidth}
            \centering
            \includegraphics[scale=.17]{figures/art-gallery-triangulated-fb-3b.pdf}          
        \end{minipage}%
        \begin{minipage}{.3\textwidth}
            \centering
            \includegraphics[scale=.17]{figures/art-gallery-triangulated-fb-4.pdf}           
        \end{minipage}
    \end{figure}

    \pagebreak

    We can color each triangle using three colors. 
    Paste these colorings together to form a 3-coloring of $P$. 
    \begin{figure}[h!]
        \begin{minipage}{.5\textwidth}
            \centering
            \includegraphics[scale=.25]{figures/art-gallery-triangulated-fb-10.pdf}      
        \end{minipage}%
        \begin{minipage}{.5\textwidth}
            \centering 
            \includegraphics[scale=.25]{figures/art-gallery-colored.pdf}          
        \end{minipage}
    \end{figure}

    We choose one of the three colors. 
    For every vertex of that color, we assign a guard. 
    Since every triangle contains that color, we know that each triangle is guarded. 
    So the whole museum can be guarded by the floor of $\frac{n}{3}$ guards.
\end{proof}

\begin{figure}[h!]
    \centering
    \includegraphics[scale=.25]{figures/art-gallery-guarded.pdf}
    % caption
\end{figure}

In our example we placed our guards at the yellow vertices, which required three guards to guard the
museum. If we had instead place guards at the green vertices then we would have only needed to use 
two guards. This helps show that the theorem is only stating how many guards will be sufficient to
guard the museum, it does not say that $\frac{n}{3}$ will be the minimum number of guards that could
guard the museum.

%%%%%%%%%%%%%%%%%%%%%%%%%%%%%%%%%%%%%%%%%%%%%%%%%%%%%%%%%%%%%%%%%%%%%%%%%%%%%%%%%%%%%%%%%%%%%%%%
%%  Applications ?
%%%%%%%%%%%%%%%%%%%%%%%%%%%%%%%%%%%%%%%%%%%%%%%%%%%%%%%%%%%%%%%%%%%%%%%%%%%%%%%%%%%%%%%%%%%%%%%%

% \section{Applications}

%%%%%%%%%%%%%%%%%%%%%%%%%%%%%%%%%%%%%%%%%%%%%%%%%%%%%%%%%%%%%%%%%%%%%%%%%%%%%%%%%%%%%%%%%%%%%%%%
%%  Conclusion
%%%%%%%%%%%%%%%%%%%%%%%%%%%%%%%%%%%%%%%%%%%%%%%%%%%%%%%%%%%%%%%%%%%%%%%%%%%%%%%%%%%%%%%%%%%%%%%%
\pagebreak

\section{Conclusion}

There are several variants to the art gallery theorem. 
For example, we may only want to guard the walls, since that's where the actual 
paintings will be, or guards may only be stationed at edges. 
An unsolved variant states: suppose each guard may patrol one wall of the museum, 
so they walk along their wall and see anything that can be seen from any point 
along that wall. How many ``wall guards'' do we then need to keep control?

\begin{figure}[h!]
    \centering
    \includegraphics[scale=.25]{figures/n4-polygon-fb.pdf}
\end{figure}

Godfried Toussaint constructed the example of a museum which shows that 
the floor of $\frac{n}{4}$ guards might be necessary. 
This polygon has 28 sides, and only requires 7 wall-guards to guard it. 
The way this is accomplished is by placing a guard at the foot of each leg 
of the polygon.
It is conjectured that, besides some small values of n, that the floor of $\frac{n}{4}$ is 
also sufficient, but there is still no proof for it.

%%%%%%%%%%%%%%%%%%%%%%%%%%%%%%%%%%%%%%%%%%%%%%%%%%%%%%%%%%%%%%%%%%%%%%%%%%%%%%%%%%%%%%%%%%%%%%%%
%%  References
%%%%%%%%%%%%%%%%%%%%%%%%%%%%%%%%%%%%%%%%%%%%%%%%%%%%%%%%%%%%%%%%%%%%%%%%%%%%%%%%%%%%%%%%%%%%%%%%

\bibliographystyle{amsplain}
\begin{thebibliography}{10}

\bibitem{AZ} M. Aigner and G. Ziegler, ``Proofs From THE BOOK,'' $3^{rd}$ edition, Springer-Verlag, 2004. 203-205.

\end{thebibliography}

\end{document}
