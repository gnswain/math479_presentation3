\documentclass{beamer}
\usetheme{Berkeley}
% \usetheme{Boadilla}
%\usetheme{Madrid}
%\usetheme{Montpellier}
%\usetheme{Warsaw}
%\usetheme{Copenhagen}
%\usetheme{Goettingen}
%\usetheme{Hannover}
%\usetheme{PaloAlto}
%\usetheme{AnnArbor}
%\usetheme{Bergen}

%\usepackage{beamerthemesplit}


\usepackage{amscd,amsxtra,amsthm}
%\usepackage[all]{xy}
%\usepackage{etex}
%\usepackage{pictex}
\usepackage{graphicx}
\usepackage{mathtools}
\usepackage{enumitem}

\theoremstyle{conjecture1}
%\newtheorem{conjecture}[theorem]{Conjecture}
\newtheorem{conjecture1}[theorem]{Conjecture 1}
\theoremstyle{conjecture2}
%\newtheorem{conjecture}[theorem]{Conjecture}
\newtheorem{conjecture2}[theorem]{Conjecture 2}

\def\G{\widetilde{G}}
\def\B{\widetilde{B}}
\def\T{\widetilde{T}}
\def\b{\widetilde{b_* }}
\def\M{\overline{M}}
\def\C{\mathbb{C}}
\def\Q{\mathbb{Q}}
\def\Z{\mathbb{Z}}
\def\F{\mathbb{F}}
\def\I{\mathbb{I}}
\def\Q{\mathbb{Q}}
\def\N{\mathbb{N}}
\def\R{\mathbb{R}}
\def\s{\mbox{\bf s}}
\def\pr{\mbox{\bf p}}
\def\i{\mbox{\bf i}}
\def\k{\mbox{\bf k}}
\def\h{\mbox{\bf h}}
\def\e{\epsilon}
\def\vp{\varpi }
\def\O{\mathcal{O}}
\def\v{\upsilon }
\def\p{\wp }
\def\z{\zeta _\upsilon}
\def\d{\cdot}
\def\c{\bullet}
\def\a{\ast}




\title{Byzantine Generals Problem}
\author[Graham Swain \\ \quad \\ Western Carolina University]{Graham Swain}
\date{October 15th, 2022 \\ Math 479}

\begin{document}

\frame{\titlepage}

%%%%%%%%%%%%%%%%%%%%%%%%%%%%%%%%%%%%%%%%%%%%%%%%%%%%%%%%%%%%%%%%%%%%%%%%%%%%%%%%%%%%%%%%%%%%%%%%
%%  Introduction
%%%%%%%%%%%%%%%%%%%%%%%%%%%%%%%%%%%%%%%%%%%%%%%%%%%%%%%%%%%%%%%%%%%%%%%%%%%%%%%%%%%%%%%%%%%%%%%%

\section{Introduction}

\begin{frame}
    \begin{enumerate}[label={\Alph*.}]
        \item All loyal generals decide upon the same plan of action.
        \item A small number of traitors cannot cause the loyal generals to adopt a bad plan.
    \end{enumerate}
\end{frame}

\begin{frame}
    \begin{itemize}[label={$\bullet$.}]
        \item Condition A is met by having the generals use the same method of decision making.
        \item<2-> Condition B is met by having the generals use a robust decision making method.
    \end{itemize}
\end{frame}

\begin{frame}
    \begin{enumerate}[label={\arabic{enumi}.}]
        \item<1-> Every loyal general must obtain the same information $v_1,...,v_n$.
        \item<2-> {
            If the $i^{th}$ general is loyal, then the value that they sends must be used by every loyal 
            general as the value of $v_i$.
        }
    \end{enumerate}
    \vspace{10pt}
    \only<4-> {Condition 1 can be rewritten as:}
    \begin{enumerate}[label={\arabic{enumi}$^\prime$.}]
        \item<4-> For every $i$, any two loyal generals use the same value of $v_i$.
    \end{enumerate} 
\end{frame}

\begin{frame}
    To help ensure that Condition A and Condition B are met, we need to meet two other conditions:
    \begin{enumerate}[label={IC\arabic{enumi}.}]
        \item<2-> All loyal lieutenants obey the same order.
        \item<3-> If the commander is loyal, then every loyal lieutenant obeys the order they sends.
    \end{enumerate} 
\end{frame}

\begin{frame}
    \begin{itemize}[label={$\bullet$.}]
        \item Loyal generals cannot take a value $v_i$ at face value.
        \item<2-> {
            Condition 1$^\prime$ and Condition 2 are both contingent on a single $v_i$ sent by the $i^{th}$
            general.
        } 
    \end{itemize}
\end{frame}

\begin{frame}
    \begin{block}{Byzantine Generals Problem}
        \only<2-> {A commanding general must send an order to their $n - 1$ lieutenants such that:}

        \only<3-> {
            \begin{enumerate}[label={IC\arabic{enumi}.}]
                \item<3-> All loyal lieutenants obey the same order.
                \item<4-> If the commander is loyal, then every loyal lieutenant obeys the order they sends.
            \end{enumerate}
        }
    \end{block}
\end{frame}

%%%%%%%%%%%%%%%%%%%%%%%%%%%%%%%%%%%%%%%%%%%%%%%%%%%%%%%%%%%%%%%%%%%%%%%%%%%%%%%%%%%%%%%%%%%%%%%%
%%  Three Generals
%%%%%%%%%%%%%%%%%%%%%%%%%%%%%%%%%%%%%%%%%%%%%%%%%%%%%%%%%%%%%%%%%%%%%%%%%%%%%%%%%%%%%%%%%%%%%%%%

\section{Three Generals}

\begin{frame}
    
\end{frame}


%%%%%%%%%%%%%%%%%%%%%%%%%%%%%%%%%%%%%%%%%%%%%%%%%%%%%%%%%%%%%%%%%%%%%%%%%%%%%%%%%%%%%%%%%%%%%%%%
%%  Oral Solution
%%%%%%%%%%%%%%%%%%%%%%%%%%%%%%%%%%%%%%%%%%%%%%%%%%%%%%%%%%%%%%%%%%%%%%%%%%%%%%%%%%%%%%%%%%%%%%%%

\section{Oral Solution}

\begin{frame}

\end{frame}


%%%%%%%%%%%%%%%%%%%%%%%%%%%%%%%%%%%%%%%%%%%%%%%%%%%%%%%%%%%%%%%%%%%%%%%%%%%%%%%%%%%%%%%%%%%%%%%%
%%  Signed Solution
%%%%%%%%%%%%%%%%%%%%%%%%%%%%%%%%%%%%%%%%%%%%%%%%%%%%%%%%%%%%%%%%%%%%%%%%%%%%%%%%%%%%%%%%%%%%%%%%

\section{Signed Solution}

\begin{frame}

\end{frame}

%%%%%%%%%%%%%%%%%%%%%%%%%%%%%%%%%%%%%%%%%%%%%%%%%%%%%%%%%%%%%%%%%%%%%%%%%%%%%%%%%%%%%%%%%%%%%%%%
%%  Missing Communication Path
%%%%%%%%%%%%%%%%%%%%%%%%%%%%%%%%%%%%%%%%%%%%%%%%%%%%%%%%%%%%%%%%%%%%%%%%%%%%%%%%%%%%%%%%%%%%%%%%

\section{Missing Communication Path}

\begin{frame}

\end{frame}


%%%%%%%%%%%%%%%%%%%%%%%%%%%%%%%%%%%%%%%%%%%%%%%%%%%%%%%%%%%%%%%%%%%%%%%%%%%%%%%%%%%%%%%%%%%%%%%%
%%  Conclusion
%%%%%%%%%%%%%%%%%%%%%%%%%%%%%%%%%%%%%%%%%%%%%%%%%%%%%%%%%%%%%%%%%%%%%%%%%%%%%%%%%%%%%%%%%%%%%%%%

\section{Conclusion}

\begin{frame}[t]

\end{frame}

%%%%%%%%%%%%%%%%%%%%%%%%%%%%%%%%%%%%%%%%%%%%%%%%%%%%%%%%%%%%%%%%%%%%%%%%%%%%%%%%%%%%%%%%%%%%%%%%
%%  References
%%%%%%%%%%%%%%%%%%%%%%%%%%%%%%%%%%%%%%%%%%%%%%%%%%%%%%%%%%%%%%%%%%%%%%%%%%%%%%%%%%%%%%%%%%%%%%%%

\frame{
    \frametitle{References}
    \begin{enumerate}{
        \scriptsize  
        \item[{[1]}] {
            L. Lamport, R. Shostak, and M. Pease, ``The Byzantine Generals Problem'', ACM Transactions on 
            Programming Languages and Systems, Vol. 4, No. 3, 1982, 382-401.
        }
    }\end{enumerate}
}

\end{document}
